% Options for packages loaded elsewhere
\PassOptionsToPackage{unicode}{hyperref}
\PassOptionsToPackage{hyphens}{url}
\PassOptionsToPackage{dvipsnames,svgnames,x11names}{xcolor}
%
\documentclass[
  letterpaper,
  DIV=11,
  numbers=noendperiod]{scrreprt}

\usepackage{amsmath,amssymb}
\usepackage{iftex}
\ifPDFTeX
  \usepackage[T1]{fontenc}
  \usepackage[utf8]{inputenc}
  \usepackage{textcomp} % provide euro and other symbols
\else % if luatex or xetex
  \usepackage{unicode-math}
  \defaultfontfeatures{Scale=MatchLowercase}
  \defaultfontfeatures[\rmfamily]{Ligatures=TeX,Scale=1}
\fi
\usepackage{lmodern}
\ifPDFTeX\else  
    % xetex/luatex font selection
\fi
% Use upquote if available, for straight quotes in verbatim environments
\IfFileExists{upquote.sty}{\usepackage{upquote}}{}
\IfFileExists{microtype.sty}{% use microtype if available
  \usepackage[]{microtype}
  \UseMicrotypeSet[protrusion]{basicmath} % disable protrusion for tt fonts
}{}
\makeatletter
\@ifundefined{KOMAClassName}{% if non-KOMA class
  \IfFileExists{parskip.sty}{%
    \usepackage{parskip}
  }{% else
    \setlength{\parindent}{0pt}
    \setlength{\parskip}{6pt plus 2pt minus 1pt}}
}{% if KOMA class
  \KOMAoptions{parskip=half}}
\makeatother
\usepackage{xcolor}
\setlength{\emergencystretch}{3em} % prevent overfull lines
\setcounter{secnumdepth}{5}
% Make \paragraph and \subparagraph free-standing
\makeatletter
\ifx\paragraph\undefined\else
  \let\oldparagraph\paragraph
  \renewcommand{\paragraph}{
    \@ifstar
      \xxxParagraphStar
      \xxxParagraphNoStar
  }
  \newcommand{\xxxParagraphStar}[1]{\oldparagraph*{#1}\mbox{}}
  \newcommand{\xxxParagraphNoStar}[1]{\oldparagraph{#1}\mbox{}}
\fi
\ifx\subparagraph\undefined\else
  \let\oldsubparagraph\subparagraph
  \renewcommand{\subparagraph}{
    \@ifstar
      \xxxSubParagraphStar
      \xxxSubParagraphNoStar
  }
  \newcommand{\xxxSubParagraphStar}[1]{\oldsubparagraph*{#1}\mbox{}}
  \newcommand{\xxxSubParagraphNoStar}[1]{\oldsubparagraph{#1}\mbox{}}
\fi
\makeatother

\usepackage{color}
\usepackage{fancyvrb}
\newcommand{\VerbBar}{|}
\newcommand{\VERB}{\Verb[commandchars=\\\{\}]}
\DefineVerbatimEnvironment{Highlighting}{Verbatim}{commandchars=\\\{\}}
% Add ',fontsize=\small' for more characters per line
\usepackage{framed}
\definecolor{shadecolor}{RGB}{241,243,245}
\newenvironment{Shaded}{\begin{snugshade}}{\end{snugshade}}
\newcommand{\AlertTok}[1]{\textcolor[rgb]{0.68,0.00,0.00}{#1}}
\newcommand{\AnnotationTok}[1]{\textcolor[rgb]{0.37,0.37,0.37}{#1}}
\newcommand{\AttributeTok}[1]{\textcolor[rgb]{0.40,0.45,0.13}{#1}}
\newcommand{\BaseNTok}[1]{\textcolor[rgb]{0.68,0.00,0.00}{#1}}
\newcommand{\BuiltInTok}[1]{\textcolor[rgb]{0.00,0.23,0.31}{#1}}
\newcommand{\CharTok}[1]{\textcolor[rgb]{0.13,0.47,0.30}{#1}}
\newcommand{\CommentTok}[1]{\textcolor[rgb]{0.37,0.37,0.37}{#1}}
\newcommand{\CommentVarTok}[1]{\textcolor[rgb]{0.37,0.37,0.37}{\textit{#1}}}
\newcommand{\ConstantTok}[1]{\textcolor[rgb]{0.56,0.35,0.01}{#1}}
\newcommand{\ControlFlowTok}[1]{\textcolor[rgb]{0.00,0.23,0.31}{\textbf{#1}}}
\newcommand{\DataTypeTok}[1]{\textcolor[rgb]{0.68,0.00,0.00}{#1}}
\newcommand{\DecValTok}[1]{\textcolor[rgb]{0.68,0.00,0.00}{#1}}
\newcommand{\DocumentationTok}[1]{\textcolor[rgb]{0.37,0.37,0.37}{\textit{#1}}}
\newcommand{\ErrorTok}[1]{\textcolor[rgb]{0.68,0.00,0.00}{#1}}
\newcommand{\ExtensionTok}[1]{\textcolor[rgb]{0.00,0.23,0.31}{#1}}
\newcommand{\FloatTok}[1]{\textcolor[rgb]{0.68,0.00,0.00}{#1}}
\newcommand{\FunctionTok}[1]{\textcolor[rgb]{0.28,0.35,0.67}{#1}}
\newcommand{\ImportTok}[1]{\textcolor[rgb]{0.00,0.46,0.62}{#1}}
\newcommand{\InformationTok}[1]{\textcolor[rgb]{0.37,0.37,0.37}{#1}}
\newcommand{\KeywordTok}[1]{\textcolor[rgb]{0.00,0.23,0.31}{\textbf{#1}}}
\newcommand{\NormalTok}[1]{\textcolor[rgb]{0.00,0.23,0.31}{#1}}
\newcommand{\OperatorTok}[1]{\textcolor[rgb]{0.37,0.37,0.37}{#1}}
\newcommand{\OtherTok}[1]{\textcolor[rgb]{0.00,0.23,0.31}{#1}}
\newcommand{\PreprocessorTok}[1]{\textcolor[rgb]{0.68,0.00,0.00}{#1}}
\newcommand{\RegionMarkerTok}[1]{\textcolor[rgb]{0.00,0.23,0.31}{#1}}
\newcommand{\SpecialCharTok}[1]{\textcolor[rgb]{0.37,0.37,0.37}{#1}}
\newcommand{\SpecialStringTok}[1]{\textcolor[rgb]{0.13,0.47,0.30}{#1}}
\newcommand{\StringTok}[1]{\textcolor[rgb]{0.13,0.47,0.30}{#1}}
\newcommand{\VariableTok}[1]{\textcolor[rgb]{0.07,0.07,0.07}{#1}}
\newcommand{\VerbatimStringTok}[1]{\textcolor[rgb]{0.13,0.47,0.30}{#1}}
\newcommand{\WarningTok}[1]{\textcolor[rgb]{0.37,0.37,0.37}{\textit{#1}}}

\providecommand{\tightlist}{%
  \setlength{\itemsep}{0pt}\setlength{\parskip}{0pt}}\usepackage{longtable,booktabs,array}
\usepackage{calc} % for calculating minipage widths
% Correct order of tables after \paragraph or \subparagraph
\usepackage{etoolbox}
\makeatletter
\patchcmd\longtable{\par}{\if@noskipsec\mbox{}\fi\par}{}{}
\makeatother
% Allow footnotes in longtable head/foot
\IfFileExists{footnotehyper.sty}{\usepackage{footnotehyper}}{\usepackage{footnote}}
\makesavenoteenv{longtable}
\usepackage{graphicx}
\makeatletter
\newsavebox\pandoc@box
\newcommand*\pandocbounded[1]{% scales image to fit in text height/width
  \sbox\pandoc@box{#1}%
  \Gscale@div\@tempa{\textheight}{\dimexpr\ht\pandoc@box+\dp\pandoc@box\relax}%
  \Gscale@div\@tempb{\linewidth}{\wd\pandoc@box}%
  \ifdim\@tempb\p@<\@tempa\p@\let\@tempa\@tempb\fi% select the smaller of both
  \ifdim\@tempa\p@<\p@\scalebox{\@tempa}{\usebox\pandoc@box}%
  \else\usebox{\pandoc@box}%
  \fi%
}
% Set default figure placement to htbp
\def\fps@figure{htbp}
\makeatother

\KOMAoption{captions}{tableheading}
\makeatletter
\@ifpackageloaded{bookmark}{}{\usepackage{bookmark}}
\makeatother
\makeatletter
\@ifpackageloaded{caption}{}{\usepackage{caption}}
\AtBeginDocument{%
\ifdefined\contentsname
  \renewcommand*\contentsname{Tabla de contenidos}
\else
  \newcommand\contentsname{Tabla de contenidos}
\fi
\ifdefined\listfigurename
  \renewcommand*\listfigurename{Listado de Figuras}
\else
  \newcommand\listfigurename{Listado de Figuras}
\fi
\ifdefined\listtablename
  \renewcommand*\listtablename{Listado de Tablas}
\else
  \newcommand\listtablename{Listado de Tablas}
\fi
\ifdefined\figurename
  \renewcommand*\figurename{Figura}
\else
  \newcommand\figurename{Figura}
\fi
\ifdefined\tablename
  \renewcommand*\tablename{Tabla}
\else
  \newcommand\tablename{Tabla}
\fi
}
\@ifpackageloaded{float}{}{\usepackage{float}}
\floatstyle{ruled}
\@ifundefined{c@chapter}{\newfloat{codelisting}{h}{lop}}{\newfloat{codelisting}{h}{lop}[chapter]}
\floatname{codelisting}{Listado}
\newcommand*\listoflistings{\listof{codelisting}{Listado de Listados}}
\makeatother
\makeatletter
\makeatother
\makeatletter
\@ifpackageloaded{caption}{}{\usepackage{caption}}
\@ifpackageloaded{subcaption}{}{\usepackage{subcaption}}
\makeatother

\ifLuaTeX
\usepackage[bidi=basic]{babel}
\else
\usepackage[bidi=default]{babel}
\fi
\babelprovide[main,import]{spanish}
% get rid of language-specific shorthands (see #6817):
\let\LanguageShortHands\languageshorthands
\def\languageshorthands#1{}
\usepackage{bookmark}

\IfFileExists{xurl.sty}{\usepackage{xurl}}{} % add URL line breaks if available
\urlstyle{same} % disable monospaced font for URLs
\hypersetup{
  pdftitle={ApLabAD},
  pdfauthor={RICUIB},
  pdflang={es},
  colorlinks=true,
  linkcolor={blue},
  filecolor={Maroon},
  citecolor={Blue},
  urlcolor={Blue},
  pdfcreator={LaTeX via pandoc}}


\title{ApLabAD}
\usepackage{etoolbox}
\makeatletter
\providecommand{\subtitle}[1]{% add subtitle to \maketitle
  \apptocmd{\@title}{\par {\large #1 \par}}{}{}
}
\makeatother
\subtitle{Apuntes de Laboratorio de Análisis de Datos}
\author{RICUIB}
\date{2025-02-01}

\begin{document}
\maketitle

\renewcommand*\contentsname{Tabla de contenidos}
{
\hypersetup{linkcolor=}
\setcounter{tocdepth}{2}
\tableofcontents
}

\bookmarksetup{startatroot}

\chapter*{Prefacio}\label{prefacio}
\addcontentsline{toc}{chapter}{Prefacio}

\markboth{Prefacio}{Prefacio}

Este libro en la web es una versión de las notas de clase de asignaturas
introductorias al análisis de datos.

\begin{description}
\tightlist
\item[Ha sido elaborado con \href{https://quarto.org/}{Quarto}]
RStudio, PBC. (2022). Quarto (Version 1.0). Hemos utilizado el formato
formato book.
\end{description}

\bookmarksetup{startatroot}

\chapter{Prerrequisitos}\label{prerrequisitos}

Para aprender cálculo de probabilidades son necesarios conocimientos de:

\begin{enumerate}
\def\labelenumi{\arabic{enumi}.}
\tightlist
\item
  Cálculo: Derivadas, integrales, límites, sumas de series\ldots{}
\item
  Geometría básica y álgebra lineal : rectas, hiperplanos,
  volúmenes\ldots{} Matrices, valores propios\ldots{}
\item
  Teoría de conjuntos y combinatoria\ldots..
\end{enumerate}

Por experiencia sabemos que la mayoría de estudiantes tienen más
conocimientos de cálculo, geometría y matrices.

Pero muchos tienen una falta de conocimientos en teoría básica de
conjuntos y combinatoria (matemática discreta).

\section{Teoría de conjuntos}\label{teoruxeda-de-conjuntos}

\textbf{Definición de conjunto}

La definición de conjunto es una
\href{https://es.wikipedia.org/wiki/Concepto_primitivo}{idea o noción
primitiva}. Es decir es una idea básica del pensamiento humano: un
conjunto es una colección de objetos: números, imágenes\ldots{}
cualquier cosa, jugadores de fútbol, palabras, colores \ldots.

La teoría de conjuntos básicas es simple y natural y es la que
necesitamos para este curso.

La teoría de conjuntos matemática es más compleja y presenta varias
paradojas como la
\href{https://es.wikipedia.org/wiki/Paradoja_de_Russell}{paradoja de
Russell}.

La idea o noción práctica de conjunto es la de una colección de objetos
de un cierto tipo.

Estas colecciones o conjuntos se pueden definir por:

\begin{itemize}
\tightlist
\item
  \green{Comprensión}: reuniendo los objetos que cumplen una propiedad
  \(p\)
\item
  \green{Extensión}: dando una lista exhaustiva de los miembros del
  conjunto
\end{itemize}

\subsection{Conjuntos básicos}\label{conjuntos-buxe1sicos}

Los conjuntos suelen tener un conjunto madre como por ejemplo

\begin{itemize}
\item
  \(\mathbb{N}=\{0,1,2,\ldots\}\)
\item
  \(\mathbb{Z}=\{\ldots,-2,-1,0,1,2,\ldots\}\)
\item
  \(\mathbb{Q}=\left\{\frac{p}{q}\quad\Big|\quad p,q\in \mathbb{Z} \mbox{ y } q \not= 0.\right\}\)
\item
  \(\mathbb{R}=\{\mbox{Todos los puntos de una recta.}\}\)
\item
  \(\mathbb{C}= \left\{a+b\cdot i\quad \big|\quad a,b\in \mathbb{R}\right\}\mbox{ los números complejos}\quad a+b\cdot i.\)
\item
  Alfabeto = \(\{a,b,c,\ldots, A,B,C,\ldots\}.\)
\item
  Palabras = \(\{paz, guerra, amor, probabilidad,\ldots\}.\)
\end{itemize}

Recordemos que \(i\) es la unidad imaginaria que cumple que
\(i=\sqrt{-1}\).

\subsection{Características y propiedades básicas de los
conjuntos}\label{caracteruxedsticas-y-propiedades-buxe1sicas-de-los-conjuntos}

\begin{itemize}
\item
  Si a cada objeto \(x\) de \(\Omega\) le llamaremos \textbf{elemento
  del conjunto} \(\Omega\) y diremos que \(x\) pertenece a \(\Omega\).
  Lo denotaremos por \(x\in \Omega\).
\item
  Un \textbf{conjunto de un elemento}, por ejemplo \(\{1\}\) recibe el
  nombre de \textbf{conjunto elemental} (o \textbf{singleton} del
  inglés).
\item
  Sea \(A\) otro conjunto diremos que \(A\) \textbf{es igual a} \(B\) si
  todos los elementos \(A\) están en \(B\) y todos los elementos de
  \(B\) están en \(A\). Por ejemplo \(A=\{1,2,3\}\) es igual a
  \(B=\{3,1,2\}\).
\item
  Si \(B\) es otro conjunto, tal que si \(x\in A\) entonces \(x\in B\)
  diremos que \(A\) es un subconjunto de o que está contenido en \(B\).
  Lo denotaremos por \(A\subseteq B.\)
\item
  El conjunto que no tiene elementos se denomina conjunto vacío y se
  denota por el símbolo \(\emptyset\).
\item
  Dado \(A\) un conjunto cualquiera obviamente \(\emptyset\subseteq A.\)
\end{itemize}

Tomemos como conjunto base \(\Omega=\{1,2,3\}\)

\begin{itemize}
\tightlist
\item
  \(\Omega\) es un conjunto de cardinal 3, se denota por
  \(\#(\Omega)=3\) o por \(|\Omega|=3\)
\item
  El conjunto \(\Omega\) tiene \(2^3=8\) subconjuntos.

  \begin{itemize}
  \tightlist
  \item
    el vacío \(\emptyset\) y los elementales \(\{1\},\{2\},\{3\}\)
  \item
    los subconjuntos de dos elementos: \(\{1,2\},\{1,3\},\{2,3\}\)
  \item
    el conjunto total de tres elementos \(\Omega=\{1,2,3\}.\)
  \end{itemize}
\end{itemize}

Dado un conjunto \(\Omega\) podemos construir el \textbf{conjunto de
todas sus partes} (todos sus subconjuntos) al que denotamos por
\(\mathcal{P}(\Omega)\). También se denomina de forma directa partes de
\(\Omega\).

Cardinal de las partes de un conjunto

El \textbf{cardinal de la partes de un conjunto} es
\(\#(\mathcal{P}(\Omega))=2^{\#(\Omega)}.\)

Por ejemplo
\(\#\left(\mathcal{P}(\{1,2,3\})\right)=2^{\#(\{1,2,3\})}=2^3=8.\)

Efectivamente

\[\mathcal{P}(\{1,2,3\})=\{\emptyset,\{1\},\{2\},\{3\},\{1,2\},\{1,3\},\{2,3\},\{1,2,3\}\}.\]

Dado un subconjunto \(A\) de \(\Omega\) podemos construir la función
característica de \(A\) \[\chi_A:\Omega \to \{0,1\}\]

dado un \(\omega\in \Omega\)

\[
\chi_A(\omega)=
\left\{
\begin{array}{ll}
1 &  \mbox{si }\omega \in A\\
0 &  \mbox{si }\omega \not\in A
\end{array}
\right.
\]

\subsection{Operaciones entre
conjuntos}\label{operaciones-entre-conjuntos}

\textbf{Intersección}

Sea \(\Omega\) un conjunto y \(A\) y \(B\) dos subconjuntos de
\(\Omega\).

El conjunto \textbf{intersección} de \(A\) y \(B\) es el formado por
todos los elementos que perteneces a \(A\) \textbf{Y} \(B\), se denota
por \(A\cap B\).

Más formalmente

\[
A\cap B=\left\{x\in\Omega \big| x\in A \mbox{ y } x\in B\right\}.
\]

\textbf{Unión.}

El conjunto \textbf{unión} de \(A\) y \(B\) es el formado por todos los
elementos que perteneces a \(A\) \textbf{O} pertenecen a \(B\), se
denota por \(A\cup B\).

Más formalmente

\[
A\cup B=\left\{x\in\Omega \big| x\in A \mbox{ o } x\in B\right\}.
\]

** Diferencia.**

El conjunto \textbf{diferencia} de \(A\) y \(B\) es el formado por todos
los elementos que perteneces a \(A\) \textbf{Y NO} pertenecen a \(B\),
se denota por \(A-B=A-(A\cap B)\).

Más formalmente

\[
A- B=\left\{x\in\Omega \big| x\in A \mbox{ y } x\notin B\right\}.
\]

** Complementario**

El \textbf{complementario} de un subconjunto \(A\) de \(\Omega\) es
\(\Omega-A\) y se denota por \(A^c\) o \(\overline{A}\).

Más formalmente

\[
A^c=\left\{x\in\Omega \big| x\not\in A\right\}.
\]

\subsection{Más propiedades}\label{muxe1s-propiedades}

Sea \(\Omega\) un conjunto y \(A\), \(B\), \(C\) tres subconjuntos de
\(\Omega\)

\begin{itemize}
\tightlist
\item
  Se dice que dos conjuntos \(A\) y \(B\) \textbf{son disjuntos} si
  \(A\cap B=\emptyset.\)
\item
  \(\Omega^c=\emptyset\).
\item
  \(\emptyset^c=\Omega\).
\item
  \(A\cup B=B \cup A\) , \(A\cap B=B\cap A\) conmutativas.
\item
  \((A\cup B) \cup C = A \cup( B \cup C)\) ,
  \((A\cap B) \cap C = A \cap( B \cap C)\) asociativas.
\item
  \(A\cup (B\cap C)=(A\cup B) \cap (A\cup C)\) ,
  \(A\cap (B\cup C)=(A\cap B) \cup (A\cap C)\) distributivas.
\item
  \(\left(A^c\right)^c=A\) doble complementario.
\item
  \(\left(A\cup B\right)^c=A^c \cap B^c\),
  \(\left(A\cap B\right)^c=A^c \cup B^c\)
  \href{https://es.wikipedia.org/wiki/Leyes_de_De_Morgan}{leyes de De
  Morgan}.
\end{itemize}

\subsection{Con R, ejemplos.}\label{con-r-ejemplos.}

Con R los conjuntos de pueden definir como vectores

\begin{Shaded}
\begin{Highlighting}[]
\NormalTok{Omega}\OtherTok{=}\FunctionTok{c}\NormalTok{(}\DecValTok{1}\NormalTok{,}\DecValTok{2}\NormalTok{,}\DecValTok{3}\NormalTok{,}\DecValTok{4}\NormalTok{,}\DecValTok{5}\NormalTok{,}\DecValTok{6}\NormalTok{,}\DecValTok{7}\NormalTok{,}\DecValTok{8}\NormalTok{,}\DecValTok{9}\NormalTok{,}\DecValTok{10}\NormalTok{)}
\NormalTok{A}\OtherTok{=}\FunctionTok{c}\NormalTok{(}\DecValTok{1}\NormalTok{,}\DecValTok{2}\NormalTok{,}\DecValTok{3}\NormalTok{,}\DecValTok{4}\NormalTok{,}\DecValTok{5}\NormalTok{)}
\NormalTok{B}\OtherTok{=}\FunctionTok{c}\NormalTok{(}\DecValTok{1}\NormalTok{,}\DecValTok{4}\NormalTok{,}\DecValTok{5}\NormalTok{)}
\NormalTok{C}\OtherTok{=}\FunctionTok{c}\NormalTok{(}\DecValTok{4}\NormalTok{,}\DecValTok{6}\NormalTok{,}\DecValTok{7}\NormalTok{,}\DecValTok{8}\NormalTok{)}
\NormalTok{Omega}
\end{Highlighting}
\end{Shaded}

\begin{verbatim}
 [1]  1  2  3  4  5  6  7  8  9 10
\end{verbatim}

\begin{Shaded}
\begin{Highlighting}[]
\NormalTok{A}
\end{Highlighting}
\end{Shaded}

\begin{verbatim}
[1] 1 2 3 4 5
\end{verbatim}

\begin{Shaded}
\begin{Highlighting}[]
\NormalTok{B}
\end{Highlighting}
\end{Shaded}

\begin{verbatim}
[1] 1 4 5
\end{verbatim}

\begin{Shaded}
\begin{Highlighting}[]
\NormalTok{C}
\end{Highlighting}
\end{Shaded}

\begin{verbatim}
[1] 4 6 7 8
\end{verbatim}

\(A\cap B\)

\begin{Shaded}
\begin{Highlighting}[]
\NormalTok{A}
\end{Highlighting}
\end{Shaded}

\begin{verbatim}
[1] 1 2 3 4 5
\end{verbatim}

\begin{Shaded}
\begin{Highlighting}[]
\NormalTok{B}
\end{Highlighting}
\end{Shaded}

\begin{verbatim}
[1] 1 4 5
\end{verbatim}

\begin{Shaded}
\begin{Highlighting}[]
\FunctionTok{intersect}\NormalTok{(A,B)}
\end{Highlighting}
\end{Shaded}

\begin{verbatim}
[1] 1 4 5
\end{verbatim}

\(A\cup B\)

\begin{Shaded}
\begin{Highlighting}[]
\NormalTok{A}
\end{Highlighting}
\end{Shaded}

\begin{verbatim}
[1] 1 2 3 4 5
\end{verbatim}

\begin{Shaded}
\begin{Highlighting}[]
\NormalTok{B}
\end{Highlighting}
\end{Shaded}

\begin{verbatim}
[1] 1 4 5
\end{verbatim}

\begin{Shaded}
\begin{Highlighting}[]
\FunctionTok{union}\NormalTok{(A,B)}
\end{Highlighting}
\end{Shaded}

\begin{verbatim}
[1] 1 2 3 4 5
\end{verbatim}

\(B-C\)

\begin{Shaded}
\begin{Highlighting}[]
\NormalTok{B}
\end{Highlighting}
\end{Shaded}

\begin{verbatim}
[1] 1 4 5
\end{verbatim}

\begin{Shaded}
\begin{Highlighting}[]
\NormalTok{C}
\end{Highlighting}
\end{Shaded}

\begin{verbatim}
[1] 4 6 7 8
\end{verbatim}

\begin{Shaded}
\begin{Highlighting}[]
\FunctionTok{setdiff}\NormalTok{(B,C)}
\end{Highlighting}
\end{Shaded}

\begin{verbatim}
[1] 1 5
\end{verbatim}

\(A^c=\Omega-A\)

\begin{Shaded}
\begin{Highlighting}[]
\NormalTok{Omega}
\end{Highlighting}
\end{Shaded}

\begin{verbatim}
 [1]  1  2  3  4  5  6  7  8  9 10
\end{verbatim}

\begin{Shaded}
\begin{Highlighting}[]
\NormalTok{A}
\end{Highlighting}
\end{Shaded}

\begin{verbatim}
[1] 1 2 3 4 5
\end{verbatim}

\begin{Shaded}
\begin{Highlighting}[]
\FunctionTok{setdiff}\NormalTok{(Omega,A)}
\end{Highlighting}
\end{Shaded}

\begin{verbatim}
[1]  6  7  8  9 10
\end{verbatim}

\subsection{Con python}\label{con-python}

\begin{Shaded}
\begin{Highlighting}[]
\NormalTok{Omega}\OperatorTok{=}\BuiltInTok{set}\NormalTok{([}\DecValTok{1}\NormalTok{,}\DecValTok{2}\NormalTok{,}\DecValTok{3}\NormalTok{,}\DecValTok{4}\NormalTok{,}\DecValTok{5}\NormalTok{,}\DecValTok{6}\NormalTok{,}\DecValTok{7}\NormalTok{,}\DecValTok{8}\NormalTok{,}\DecValTok{9}\NormalTok{,}\DecValTok{10}\NormalTok{])}
\NormalTok{A}\OperatorTok{=}\BuiltInTok{set}\NormalTok{([}\DecValTok{1}\NormalTok{,}\DecValTok{2}\NormalTok{,}\DecValTok{3}\NormalTok{,}\DecValTok{4}\NormalTok{,}\DecValTok{5}\NormalTok{])}
\NormalTok{B}\OperatorTok{=}\BuiltInTok{set}\NormalTok{([}\DecValTok{1}\NormalTok{,}\DecValTok{4}\NormalTok{,}\DecValTok{5}\NormalTok{])}
\NormalTok{C}\OperatorTok{=}\BuiltInTok{set}\NormalTok{([}\DecValTok{4}\NormalTok{,}\DecValTok{6}\NormalTok{,}\DecValTok{7}\NormalTok{,}\DecValTok{8}\NormalTok{])}
\NormalTok{Omega}
\end{Highlighting}
\end{Shaded}

\begin{verbatim}
{1, 2, 3, 4, 5, 6, 7, 8, 9, 10}
\end{verbatim}

\begin{Shaded}
\begin{Highlighting}[]
\NormalTok{A}
\end{Highlighting}
\end{Shaded}

\begin{verbatim}
{1, 2, 3, 4, 5}
\end{verbatim}

\begin{Shaded}
\begin{Highlighting}[]
\NormalTok{B}
\end{Highlighting}
\end{Shaded}

\begin{verbatim}
{1, 4, 5}
\end{verbatim}

\begin{Shaded}
\begin{Highlighting}[]
\NormalTok{C}
\end{Highlighting}
\end{Shaded}

\begin{verbatim}
{8, 4, 6, 7}
\end{verbatim}

\begin{Shaded}
\begin{Highlighting}[]
\NormalTok{A }\OperatorTok{\&}\NormalTok{ B   }\CommentTok{\# intersección (\&: and/y)}
\end{Highlighting}
\end{Shaded}

\begin{verbatim}
{1, 4, 5}
\end{verbatim}

\begin{Shaded}
\begin{Highlighting}[]
\NormalTok{A }\OperatorTok{|}\NormalTok{ B   }\CommentTok{\# unión (|: or/o)}
\end{Highlighting}
\end{Shaded}

\begin{verbatim}
{1, 2, 3, 4, 5}
\end{verbatim}

\begin{Shaded}
\begin{Highlighting}[]
\NormalTok{A }\OperatorTok{{-}}\NormalTok{ C   }\CommentTok{\# diferencia }
\end{Highlighting}
\end{Shaded}

\begin{verbatim}
{1, 2, 3, 5}
\end{verbatim}

\begin{Shaded}
\begin{Highlighting}[]
\NormalTok{Omega}\OperatorTok{{-}}\NormalTok{C }\CommentTok{\# complementario.}
\end{Highlighting}
\end{Shaded}

\begin{verbatim}
{1, 2, 3, 5, 9, 10}
\end{verbatim}

\section{Combinatoria}\label{combinatoria}

La combinatoria es una rama de la matemática discreta que entre otras
cosas cuenta distintas configuraciones de objetos de un conjunto.

Por ejemplo si tenemos un equipo de baloncesto con 7 jugadores ¿cuántos
equipos de 5 jugadores distintos podemos formar?

\subsection{Número Binonial}\label{nuxfamero-binonial}

\textbf{Número combinatorio o número binomial}

Nos da el número de subconjuntos de tamaño \(k\) de un conjunto de
tamaño \(n\). Este número es

\[
C_n^k={n\choose k} = \frac{n!}{k!\cdot (n-k)!}.
\]

Recordemos que \[
n!=1\cdot 2\cdot 3\cdots n.
\]

En nuestro caso con 7 jugadores \(n=7\) el número de equipos distintos
de \(k=5\) es

\[
\begin{array}{rl}
C_7^5&={7\choose 5} = \frac{7!}{5!\cdot (7-5)!}=\frac{7!}{5!\cdot 2!} \\
&=\frac{1\cdot 2\cdot 3 \cdot 4\cdot 5\cdot 6\cdot 7}{1\cdot 2\cdot 3 \cdot 4\cdot 5\cdot 1\cdot 2}=\frac{6\cdot 7}{2}=\frac{42}{2}=21.
\end{array}
\]

Puedo formar 21 equipos distintos.

\textbf{Ejercicio}

Carga el paquete \texttt{gtools} de R y investiga la función
\texttt{combinations(n,\ r,\ v,\ set,\ repeats.allowed)} para calcular
todas las combinaciones anteriores.

\subsection{Combinaciones con
repetición}\label{combinaciones-con-repeticiuxf3n}

En combinatoria, las combinaciones con repetición de un conjunto son las
distintas formas en que se puede hacer una selección de elementos de un
conjunto dado, permitiendo que las selecciones puedan repetirse.

El número \(CR_n^k\) de multiconjuntos con \(k\) elementos escogidos de
un conjunto con \(n\) elementos satisface:

\begin{itemize}
\tightlist
\item
  Es igual al número de combinaciones con repetición de \(k\) elementos
  escogidos de un conjunto con \(n\) elementos.
\item
  Es igual al número de formas de repartir \(k\) objetos en \(n\)
  grupos.
\end{itemize}

\[CR_n^k = \binom{n+k-1}{k} = \frac{(n+k-1)!}{k!(n-1)!}.\]

\textbf{Ejemplo}

Vamos a imaginar que vamos a repartir 12 caramelos entre Antonio,
Beatriz, Carlos y Dionisio (que representaremos como A, B, C, D). Una
posible forma de repartir los caramelos sería: dar 4 caramelos a
Antonio, 3 a Beatriz, 2 a Carlos y 3 a Dionisio. Dado que no importa el
orden en que se reparten, podemos representar esta selección como
AAAABBBCCDDD.

Otra forma posible de repartir los caramelos podría ser: dar 1 caramelo
a Antonio, ninguno a Beatriz y Carlos, los 11 restantes se los damos a
Dionisio. Esta repartición la representamos como ADDDDDDDDDDD

Recíprocamente, cualquier serie de 12 letras A, B, C, D se corresponde a
una forma de repartir los caramelos. Por ejemplo, la serie AAAABBBBBDDD
corresponde a: Dar 4 caramelos a Antonio, 5 caramelos a Beatriz, ninguno
a Carlos y 3 a Dionisio.

De esta forma, el número de formas de repartir los caramelos es:

\[CR_{n=4}^{k=12} = \binom{4+12-1}{12}=455.\]

\subsection{Variaciones.}\label{variaciones.}

\textbf{Variaciones}

Con los número \(\{1,2,3\}\) ¿cuántos números de dos cifras distintas
podemos formar sin repetir ninguna cifra?

La podemos escribir

\[12,13,21,23,31,32\]

Luego hay seis casos

Denotaremos las variaciones (sin repetición) de \(k\) elementos (de
orden \(k\)) de un conjunto de \(n\) elementos por \(V_n^k\) su valor es

\[
V_n^k=\frac{n!}{(n-k)!}=(n-k+1)\cdot (n-k+2)\cdots n.
\]

En nuestro ejemplo con \(n=3\) dígitos podemos escribir las siguientes
variaciones de orden \(k=2\)

\[
V^{k=2}_{n=3}=\frac{3!}{(3-2)!}=\frac{1\cdot 2\cdot 3}{1}=6.
\]

\textbf{Ejercicio}

Carga el paquete \texttt{gtools} de R y investiga la función
\texttt{permutations(n,\ r,\ v,\ set,\ repeats.allowed)} para calcular
todas las variaciones anteriores.

\subsection{Variaciones con
repetición.}\label{variaciones-con-repeticiuxf3n.}

\textbf{Variaciones con repetición}

¿Y repitiendo algún dígito?

\[VR_n^k=n^k\]

Efectivamente en nuestro caso

\[11,12,13,21,22,23,31,32,33\]

\[
VR^{k=2}_{n=3}=n^k.
\]

\subsection{Permutaciones}\label{permutaciones}

Las permutaciones de un conjunto de cardinal \(n\) son todas las
variaciones de orden máximo \(n\). Las denotamos y valen:

\[
P_n=V_n^n=n!
\]

Por ejemplo todos los números que se pueden escribir ordenando todos los
dígitos \(\{1,2,3\}\) sin repetir ninguno

\begin{Shaded}
\begin{Highlighting}[]
\FunctionTok{library}\NormalTok{(combinat)}
\ControlFlowTok{for}\NormalTok{(permutacion }\ControlFlowTok{in} \FunctionTok{permn}\NormalTok{(}\DecValTok{3}\NormalTok{)) }\FunctionTok{print}\NormalTok{(permutacion)}
\end{Highlighting}
\end{Shaded}

\begin{verbatim}
[1] 1 2 3
[1] 1 3 2
[1] 3 1 2
[1] 3 2 1
[1] 2 3 1
[1] 2 1 3
\end{verbatim}

Efectivamente \(P_3=3!=1\cdot  2\cdot 3.\)

\textbf{Ejercicio}

Carga el paquete \texttt{combinat} de R e investiga la función
\texttt{permn} para calcular todas las permutaciones anteriores.

\textbf{Ejercicio}

Investiga el paquete \texttt{itertools} y la función \texttt{comb} de
\texttt{scipy.misc} de Python e investiga sus funciones para todas las
formas de contar que hemos visto en este tema.

\textbf{Ejercicio}

La función gamma de Euler, cobrará mucha importancia en el curso de
estadística. Comprueba que la función \texttt{gamma(x+1)} da el mismo
valor que la función \texttt{factorial(x)} en \texttt{R} para todo
\(x = \{1,2,3,\ldots,10\}\).

\subsection{Números multinomiales. Permutaciones con
repetición.}\label{nuxfameros-multinomiales.-permutaciones-con-repeticiuxf3n.}

Consideremos un conjunto de elementos \(\{a_1, a_2, \ldots, a_k\}\).

Entonces, si cada uno de los objetos \(a_i\) de un conjunto, aparece
repetido \(n_i\) veces para cada \(i\) desde 1 hasta \(k\), entonces el
número de permutaciones con elementos repetidos es:

\[PR_n^{n_1,n_2,\ldots,n_k} = {{n}\choose {n_1\quad n_2 \quad\ldots \quad n_k}}=\frac{n!}{n_1!\cdot n_2!\cdot \ldots \cdot n_k!},\]
donde \(n=n_1+n_2+\cdots+n_k\).

\textbf{Ejemplo}

¿Cuantas palabras diferentes se pueden formar con las letras de la
palabra \texttt{PROBABILIDAD}?

El conjunto de letras de la palabra considerada es el siguiente:
\(\{A, B, D, I, L, O, P, R\}\) con las repeticiones siguientes: las
letras A, B, D, e I, aparecen 2 veces cada una; y las letras L, O, P, R
una vez cada una de ellas.

Por tanto, utilizando la fórmula anterior, tenemos que el número de
palabras (permutaciones con elementos repetidos) que podemos formar es

\[PR^{2,2,2,2,1,1,1,1}_{12} = \frac{12!}{(2!)^4(1!)^4} = 29937600.\]

\bookmarksetup{startatroot}

\chapter{Para acabar}\label{para-acabar}

\section{Principios básicos para contar cardinales de
conjuntos}\label{principios-buxe1sicos-para-contar-cardinales-de-conjuntos}

\textbf{El principio de la suma}

Sean \(A_1, A_2,\ldots, A_n\) conjuntos disjuntos dos a dos, es decir
\(A_i\cap A_j=\emptyset\) para todo \(i\not= j\), \(i,j=1,2,\ldots n\).
Entonces

\[\#(\cup_{i=1}^n A_i)=\sum_{i=1}^n \#(A_i).\]

\textbf{Principio de unión exclusión}

Consideremos dos conjuntos cualesquiera \(A_1, A_2\) entonces el
cardinal de su unión es

\[\#(A_1\cup A_2)=\#(A_1)+\#(A_2)-\#(A_1\cap A_2).\]

\textbf{El principio del producto}

Sean \(A_1,A_2,\ldots A_n\)

\[
\begin{array}{ll}
\#(A_1\times A_2\times \cdots A_n)=&\#\left(\{(a_1,a_2,\ldots a_n)| a_i\in A_i, i=1,2,\ldots n\}\right)\\
&=\prod_{i=1}^n \#(A_i).
\end{array}
\]

\section{Otros aspectos a tener en
cuenta}\label{otros-aspectos-a-tener-en-cuenta}

Evidentemente nos hemos dejado muchas otras propiedades básicas de
teoría de conjuntos y de combinatoria como:

\begin{itemize}
\tightlist
\item
  Propiedades de los números combinatorios.
\item
  Binomio de Newton.
\item
  Multinomio de Newton.
\end{itemize}

Si nos son necesarias las volveremos a repetir a lo largo del curso o
bien daremos enlaces para que las podáis estudiar en paralelo.

\part{Probabilidad y variables aleatorias}

\chapter{Preliminares: conjuntos y
combinatoria}\label{preliminares-conjuntos-y-combinatoria}

\chapter{Probabilidad}\label{probabilidad}

\chapter{Variables alestorias}\label{variables-alestorias}

\chapter{Distribuciones notables 1}\label{distribuciones-notables-1}

\chapter{Distribuciones notables 2}\label{distribuciones-notables-2}




\end{document}
